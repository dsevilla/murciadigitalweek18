\PassOptionsToPackage{unicode}{hyperref}
\PassOptionsToPackage{naturalnames}{hyperref}
\PassOptionsToPackage{table}{xcolor}

\documentclass[14pt]{beamer}

\mode<presentation> {

% The Beamer class comes with a number of default slide themes
% which change the colors and layouts of slides. Below this is a list
% of all the themes, uncomment each in turn to see what they look like.

%\usetheme{default}
%\usetheme{AnnArbor}
%\usetheme{Antibes}
%\usetheme{Bergen}
%\usetheme{Berkeley}
%\usetheme{Berlin}
%\usetheme{Boadilla}
%\usetheme{CambridgeUS}
%\usetheme{Copenhagen}
%\usetheme{Darmstadt}
%\usetheme{Dresden}
%\usetheme{Frankfurt}
%\usetheme{Goettingen}
%\usetheme{Hannover}
%\usetheme{Ilmenau}
%\usetheme{JuanLesPins}
%\usetheme{Luebeck}
%\usetheme{Madrid}
%\usetheme{Malmoe}
%\usetheme{Marburg}
%\usetheme{Montpellier}
%\usetheme{PaloAlto}
%\usetheme{Pittsburgh}
%\usetheme{Rochester}
%\usetheme{Singapore}
%\usetheme{Szeged}
%\usetheme{Warsaw}

% As well as themes, the Beamer class has a number of color themes
% for any slide theme. Uncomment each of these in turn to see how it
% changes the colors of your current slide theme.

%\usecolortheme{albatross}
%\usecolortheme{beaver}
%\usecolortheme{beetle}
%\usecolortheme{crane}
%\usecolortheme{dolphin}
%\usecolortheme{dove}
%\usecolortheme{fly}
%\usecolortheme{lily}
%\usecolortheme{orchid}
%\usecolortheme{rose}
%\usecolortheme{seagull}
%\usecolortheme{seahorse}
%\usecolortheme{whale}
%\usecolortheme{wolverine}

%\setbeamertemplate{footline} % To remove the footer line in all slides uncomment this line
%\setbeamertemplate{footline}[page number] % To replace the footer line in all slides with a simple slide count uncomment this line

\setbeamertemplate{navigation symbols}{} % To remove the navigation symbols from the bottom of all slides uncomment this line
\setbeamertemplate{frametitle continuation}[from second][(\insertcontinuationcountroman)]
}

\usepackage{graphicx} % Allows including images
\usepackage{booktabs} % Allows the use of \toprule, \midrule and \bottomrule in tables
% \usepackage[T1]{fontenc}
\usepackage[spanish]{babel}
\usepackage[utf8]{inputenc}
\usepackage{listings}
\usepackage{tikz}
\usetikzlibrary{calc,math,shapes.callouts,shapes.arrows,shapes.geometric,fit,positioning,arrows.meta}
\usetikzlibrary{positioning,backgrounds}
\usetikzlibrary{decorations.pathmorphing,math}
%\usepackage{iwona}
%\usepackage{marvosym}
%\usepackage{cfr-lm}
%\usepackage{pifont}
%\usepackage{keystroke}
%\usepackage{etoolbox}


\definecolor{mdwrojo}{HTML}{CC1A4C}
\setbeamercolor{titlelike}{fg=mdwrojo} % titulo
\setbeamercolor{frametitle}{fg=mdwrojo} % titulo
\setbeamercolor{section in toc}{fg=mdwrojo} % titulo
\setbeamercolor{subsection in toc}{fg=mdwrojo} % titulo
% https://tex.stackexchange.com/questions/185742/i-need-to-change-color-of-beamer-itemize-and-subitem-separately
%\setbeamercolor{itemize item}{fg=mdwrojo}
\setbeamercolor{itemize subitem}{fg=mdwrojo}
\setbeamercolor{block title}{bg=mdwrojo,fg=white}
\setbeamercolor{block body}{bg=mdwrojo!20}

%\setbeamercolor{block title}{bg=blue!30,fg=black}
%\setbeamercolor{block body}{bg=blue!20,fg=black}
%\setbeamercolor{block title alerted}{bg=red!30,fg=black}
%\setbeamercolor{block body alerted}{bg=red!20,fg=black}

\newcommand*{\murciabullet}{\kern1em\includegraphics[height=1em]{img/mdw-bullet}}
\setbeamertemplate{itemize item}{\murciabullet}


% Language Definitions for CYPHER
\lstdefinelanguage{cypher}{
morecomment=[l][\color{gray}]{//},
morestring=[b][\color{olive}]\",
morestring=[b][\color{olive}]\',
morekeywords={MATCH,WHERE,LIMIT,CREATE,RETURN,DISTINCT,DELETE,DETACH,UNIQUE,MERGE,INDEX,ON,SET,LOAD,CSV,FOREACH,IN},
sensitive=true
}

%
% Listados de código
%
\lstset{%
basicstyle=\ttfamily\footnotesize,
commentstyle=\color{gray}\itshape\ttfamily,
keywordstyle=\color{blue!80}\bfseries\ttfamily,
stringstyle = \color{gray},
showstringspaces=false,
frame=tblr, % single, tb, ltrb % boxed listings, en mayusculas = doble linea
framerule=0pt,
tabsize=4, % tabulador = 2 espacios
captionpos=b,
backgroundcolor=\color{mdwrojo!20},
breaklines=true,
%backgroundcolor=\color{white},
%numbers=left, numberstyle=\tiny, stepnumber=2, numbersep=10pt,
xleftmargin=0.02\textwidth,
xrightmargin=0.02\textwidth,
language=java, % Por defecto
literate={«}{{\guillemotleft}}1
           {»}{{\guillemotright}}1
{á}{{\'a}}1 {é}{{\'e}}1 {í}{{\'i}}1 {ó}{{\'o}}1 {ú}{{\'u}}1
  {Á}{{\'A}}1 {É}{{\'E}}1 {Í}{{\'I}}1 {Ó}{{\'O}}1 {Ú}{{\'U}}1
  {à}{{\`a}}1 {è}{{\`e}}1 {ì}{{\`i}}1 {ò}{{\`o}}1 {ù}{{\`u}}1
  {À}{{\`A}}1 {È}{{\'E}}1 {Ì}{{\`I}}1 {Ò}{{\`O}}1 {Ù}{{\`U}}1
  {ä}{{\"a}}1 {ë}{{\"e}}1 {ï}{{\"i}}1 {ö}{{\"o}}1 {ü}{{\"u}}1
  {Ä}{{\"A}}1 {Ë}{{\"E}}1 {Ï}{{\"I}}1 {Ö}{{\"O}}1 {Ü}{{\"U}}1
  {â}{{\^a}}1 {ê}{{\^e}}1 {î}{{\^i}}1 {ô}{{\^o}}1 {û}{{\^u}}1
  {Â}{{\^A}}1 {Ê}{{\^E}}1 {Î}{{\^I}}1 {Ô}{{\^O}}1 {Û}{{\^U}}1
  {œ}{{\oe}}1 {Œ}{{\OE}}1 {æ}{{\ae}}1 {Æ}{{\AE}}1 {ß}{{\ss}}1
  {ű}{{\H{u}}}1 {Ű}{{\H{U}}}1 {ő}{{\H{o}}}1 {Ő}{{\H{O}}}1
  {ç}{{\c c}}1 {Ç}{{\c C}}1 {ø}{{\o}}1 {å}{{\r a}}1 {Å}{{\r A}}1
  {€}{{\euro}}1 {£}{{\pounds}}1
           {ñ}{{\~n}}1
           {Ñ}{{\~N}}1
           {¿}{{?`}}1
}

\colorlet{punct}{red!60!black}
\definecolor{background}{HTML}{EEEEEE}
\definecolor{delim}{RGB}{20,105,176}
\colorlet{numb}{magenta!60!black}

\lstdefinelanguage{json}{
    basicstyle=\ttfamily,
    stepnumber=1,
    numbersep=8pt,
    showstringspaces=false,
    breaklines=true,
%    frame=lines,
    moredelim=**[is][\color{red}]{@}{@},
    moredelim=**[is][\color{blue}]{º}{º},
%    backgroundcolor=\color{background},
    literate=
      {:}{{{\color{punct}{:}}}}{1}
      {,}{{{\color{punct}{,}}}}{1}
      {\{}{{{\color{delim}{\{}}}}{1}
      {\}}{{{\color{delim}{\}}}}}{1}
      {[}{{{\color{delim}{[}}}}{1}
      {]}{{{\color{delim}{]}}}}{1}
}

\definecolor{darkgray}{rgb}{.4,.4,.4}
\definecolor{purple}{rgb}{0.65, 0.12, 0.82}

\lstdefinelanguage{JavaScript}
{
  basicstyle=\ttfamily,
  keywords={typeof,new,true,false,catch,function,return,null,catch,switch,var,if,in,while,do,else,case
,break},
ndkeywords={class, export, boolean, throw, implements, import, this},
ndkeywordstyle=\color{darkgray}\bfseries,
sensitive=false,
comment=[l]{//},
morecomment=[s]{/*}{*/},
morestring=[b]',
morestring=[b]"
}

%% Macros comunes
\newcommand{\hide}[1]{}
\newcommand{\ra}{{\color{mdwrojo} $\Rightarrow${}~{}}}

\usepackage{fontspec}
\defaultfontfeatures{Ligatures=TeX,Numbers=OldStyle}
% \setmainfont{Aller_Lt.ttf}[
% BoldFont = Aller_Rg.ttf,
% ItalicFont = Aller_LtIt.ttf,
% BoldItalicFont = Aller_It.ttf]
\setsansfont
  [Ligatures=TeX, % recommended
   UprightFont={* Light},
   ItalicFont={* Light Italic},
   BoldFont={*},
   BoldItalicFont={* Italic}]
   {Roboto}
   % {Open Sans}
% \setsansfont
%   [Ligatures=TeX, % recommended
%    UprightFont={* Regular},
%    ItalicFont={* Italic}]
%   {Fira Sans}
%\setmainfont{Open Sans}[BoldFont={* Bold}]
% \setmonofont[Ligatures = TeX,
% UprightFont={* Light},
%    ItalicFont={* Light Italic},
%    BoldFont={* Medium},
%    BoldItalicFont={* Medium Italic}]{Input Mono}

%\setmonofont{Input Mono}
\setmonofont%[Scale=1.1]
 % [Ligatures=TeX, % recommended
 %  UprightFont={* Regular},
 %  ItalicFont={* Italic},
 %  BoldFont={* Bold},
 %  BoldItalicFont={* Bold Italic}]
%{Fira Mono}
{Source Code Pro}

\newsavebox{\mysubpic}

\usebackgroundtemplate{
%\setbox{\mysubpic}{
  \sbox{\mysubpic}{%
    \begin{tikzpicture}[remember picture,line width=1em,mdwrojo!30] %sub-picture
      \foreach \s in {1,...,\value{framenumber}}{
        \tikzmath{
          int \shift;
          \shift = (\s * 2);
          if Mod(\s,5) > 0 then {
            { \draw[xshift=\shift em] (0,0) -- (0,5em); };
          } else {
            { \draw[xshift=\shift em] (-.5em,3em) -- (-9.5em,1em); };
          };
       }
      }
    \end{tikzpicture}% needed, otherwise anchors are wrong!
  }

  \begin{tikzpicture}[remember picture,overlay,line width=2em]
    %\node[opacity=0.3, at=(current page.south east),anchor=south east,inner sep=0pt] {
%    \includegraphics[height=\paperheight,width=\paperwidth]{image}};
    \coordinate[at=(current page.north west)] (ul);
    \coordinate[at=(current page.south west)] (sw);
    \coordinate[at=(current page.south east)] (lr);
%    \path (ul) -- (lr) node[opacity=0.3,midway,anchor=center]{\usebox{\mysubpic}};
    \path (sw) -- (lr) node[opacity=.7,pos=.99,anchor=south east,scale=.1]{\usebox{\mysubpic}};
  \end{tikzpicture}
}

\hypersetup{%
  pdftitle={Big Data desde un punto de vista tecnológico},%
  pdfauthor={Diego Sevilla Ruiz},
  pdfsubject={Big Data, MDW'18},
  pdfkeywords={big data, mdw18}
}


%----------------------------------------------------------------------------------------
%       TITLE PAGE
%----------------------------------------------------------------------------------------

\title{Big Data:\\Un punto de vista
  tecnológico\thanks{\url{https://github.com/dsevilla/murciadigitalweek18}}}
\subtitle{MDW, 2018}

\author{Diego Sevilla Ruiz}
\institute[UMU]
{
Dpto. de Ingeniería y Tecnología de Computadores\\
Facultad de Informática\\
Universidad de Murcia\\
\medskip
\href{mailto:dsevilla@um.es}{\texttt{dsevilla@um.es}}
}
\date{Junio de 2018}

\makeatletter
\patchcmd{\beamer@sectionintoc}{\vskip1.5em}{\vskip1em}{}{}
\makeatother

\begin{document}

%\def\insertsectionnumber{\arabic{section}}

% \AtBeginSection[]{

% \begin{frame}[plain]

%   \begin{centering}
%     \begin{beamercolorbox}[sep=10pt,center]{part title}
%       {\huge \bf \textcolor{white}{\insertsectionnumber.~\insertsection}}
%     \end{beamercolorbox}
%   \end{centering}

%   \end{frame}
% }

%\def\insertsubsectionnumber{\arabic{subsection}}

% \AtBeginSection[]{
%   \begin{frame}<beamer>
%     \frametitle{\insertsubsection}

%     \tableofcontents[currentsection]
%   \end{frame}
% }
% \AtBeginSubsection[]{
%   \begin{frame}<beamer>
%     \frametitle{\insertsubsection}

%     \tableofcontents[currentsection,currentsubsection]
%   \end{frame}
% }
{
  \usebackgroundtemplate{%
    \vbox to \paperheight{\vfil\includegraphics[width=\paperwidth]{img/mdw}\vfil}}
\begin{frame}
\titlepage % Print the title page as the first slide
\end{frame}
}

\section{Introducción}

\begin{frame}
  \frametitle{Big Data}
  \framesubtitle{\url{https://twitter.com/jmibl/status/390768769259163648}.}
\centering\includegraphics[width=.9\textwidth]{img/big_data_sex}
\end{frame}

\begin{frame}
  \frametitle{Big Data}
  \centering\includegraphics[width=\textwidth]{img/bigdata1}
\end{frame}

\begin{frame}
  \frametitle{Big Data -- 3Vs}
  \framesubtitle{\url{https://www.theviable.co/how-big-data-impact-to-corporate/3v-model-of-big-data/}}
\centering\includegraphics[width=\textwidth]{img/3v}
\end{frame}

\begin{frame}
  \frametitle{Big Data -- ¡¡¡8Vs!!!}
  \centering\includegraphics[width=.8\textwidth]{img/8V}
\end{frame}

\begin{frame}
  \frametitle{Big Data Landscape 2017}
  \framesubtitle{\url{http://mattturck.com/bigdata2017/}}
  \centering\includegraphics[width=\textwidth]{img/Big-Data-Landscape}
\end{frame}

\begin{frame}
  \frametitle{Big Data -- Un minuto del día}
  \framesubtitle{\url{https://www.domo.com/learn/data-never-sleeps-6}}
  \centering\includegraphics[height=.8\textheight]{img/data-never-sleeps-6}
\end{frame}

\begin{frame}
  \frametitle{Big Data -- Un minuto del día}
  \framesubtitle{\url{https://www.domo.com/learn/data-never-sleeps-6}}
  \centering\includegraphics[height=.8\textheight]{img/data-never-sleeps-6-2}
\end{frame}

\begin{frame}
  \frametitle{Big Data -- ¡Y eso sin contar IoT!}
  \centering\includegraphics[width=\textwidth]{img/iot}
\end{frame}

\begin{frame}
  \frametitle{Big Data -- Airbus A350}
  \framesubtitle{\url{https://siliconsemiconductor.net/article/102842/Aviation_depends_on_sensors_and_big_data}}
  \centering\includegraphics[width=\textwidth]{img/a350}
\end{frame}

\begin{frame}
  \frametitle{Big Data, entonces...}
  \begin{itemize}
  \item Recopilar todos los datos posibles
    \begin{itemize}
    \item Después serán útiles, se podrán analizar
    \item El coste de almacenamiento cada vez es menor
    \item Si no \ra{} coste de oportunidad
    \end{itemize}
  \item {\bfseries\itshape Data Science}
    \begin{itemize}
    \item El análisis ofrecerá {\bfseries\itshape conocimiento} para
      mejorar ({\bf valor})
    \end{itemize}
  \item Imposible procesar todo {\bfseries\itshape online}:
    \begin{itemize}
    \item Separación entre capa {\em batch\/} y capa {\em online}
    \item Lambda Architecture
    \end{itemize}
  \end{itemize}
\end{frame}

\begin{frame}
  \frametitle{Coste por GB}
  \centering\includegraphics[width=\textwidth]{img/cost-per-gigabyte}
\end{frame}

\begin{frame}
  \frametitle{Lambda Architecture}
  \framesubtitle{\url{https://mapr.com/developercentral/lambda-architecture/}}
  \centering\includegraphics[width=\textwidth]{img/lambda-architecture}
\end{frame}

\section{Data Science}

\begin{frame}
  \frametitle{Data Science}
  \framesubtitle{Howe, 2013}
\vspace*{-.5em}
  \begin{quote}
    ``Next sexy job'' \\
    ``The ability to take data—to be able to understand it, to
process it, to extract value from it, to visualize it, to
communicate it—that’s going to be a hugely important skill.'' \\
  \hspace*\fill{\small--- Hal Varian, Google}
  \end{quote}
    \begin{quote}
``Data science is the civil engineering of data. Its acolytes
possess a practical knowledge of tools \& materials, coupled
with a theoretical understanding of what's possible.''\\
  \hspace*\fill{\small--- Mike Driscoll, Metamarkets}
  \end{quote}

\end{frame}

\begin{frame}
  \frametitle{Data Science}
  \framesubtitle{\url{http://drewconway.com/zia/2013/3/26/the-data-science-venn-diagram}}
\begin{center}
Drew Conway Venn Diagram of Big Data:
\includegraphics[width=.5\textwidth]{img/Data_Science_VD}
\end{center}
\end{frame}

\begin{frame}
  \frametitle{Data Science}

  Mike Driscoll’s three sexy skills of data geeks

  \begin{itemize}
  \item Statistics
    \begin{itemize}
    \item traditional analysis
\end{itemize}
\item Data Munging
  \begin{itemize}
  \item parsing, scraping, and formatting data
\end{itemize}
\item Visualization
  \begin{itemize}
  \item graphs, tools, etc.
\end{itemize}
\end{itemize}

\begin{block}{}
  \begin{center}
    (data wrangling, data jujitsu, data munging)
  \end{center}
\end{block}
\end{frame}

\begin{frame}[plain]
%  \frametitle{Data Science}
  \includegraphics[width=\textwidth]{img/gray-grep}
\end{frame}


% [MUSIC] Welcome back. So I wanna talk a little bit about how the term
% Data Science relates to other fields of science. And, in particular, I
% wanna introduce this term, eScience, which to a first proximation you can
% think of is equivalent to Data Science. So while the term eScience is
% associated with astronomy and oceanography and biology, Data Science has
% been adopted more in business, but they involve a lot of the same
% concepts. So let me tell you about what's going on in science. So for
% thousands of years, scientific inquiry has been empirical, right? You
% observe the natural world, or in some cases maybe replicate the natural
% world, in a controlled environment in the laboratory and make
% observations about that. In the last few hundreds of years, science has
% accepted theoretical models as a valid method of inquiry, one that is
% reinforcing empirical methods. So, new theories suggest new experiments,
% and the theories help explain the observed data you get from the
% experiments. In the last 50 years or so, high speed computation has
% emitted an entirely new method of scientific inquiry. You can simulate in
% the computer phenomena that otherwise you can't observe directly and you
% can't reproduce in the lab. And even the theoretical models become too
% complex to solve analytically, using essentially paper and pencil, right?
% But you can actually start from initial conditions and run the simulation
% to get a result. So this is, maybe, the what goes on in the interior of
% stars, or the shift of tectonic plates, or the evolution of the universe,
% or the effects on the ecology from some species dying out, and so on. So
% that's fine, so that's three methods of inquiry. But in the last 10 years
% or so, there's been, arguably, a fourth method of scientific inquiry,
% which is to acquire massive data sets from instruments or from
% simulations, and then explore these data sets using new algorithms and
% new infrastructure. And so eScience is really about massive and complex
% data, data large enough to require automated or semi-automated analysis.
% You can't look at it, you can't inspect it directly. Okay. And so the
% relevant tools here are the same as those for data science, databases,
% visualization, scale out computing, maybe the NoSQL systems, machine
% learning techniques, web services, and so on. This idea of the fourth
% paradigm, there's a book that's in the reading list that you can refer to
% here. And there's some other articles in the reading list that you can
% also refer to. The story's been told lots of ways. The way I like to talk
% about this story is that science has always been about asking questions,
% but conventionally it was really about querying the world, right? You
% would sort of have data acquisition activities, experiments or field
% studies, that were coupled to very specific hypotheses, right? You had
% the question in mind, first, and went out and collected data. But
% eScience has really sort of shifted a bit where now you're downloading
% data en masse, you're downloading the world first, putting some sort of a
% representation into the computer and then querying that database to test
% your hypotheses. And so the data can be acquired independent of any
% specific hypothesis in some cases. Okay. And this is due in part to the
% cost of data acquisition dropping precipitously thanks to advances in
% technology, right, so the telescopes you can build now that we'll talk
% about in the next couple of slides can acquire enormous amounts of data
% at very high resolution. Okay. And in the life sciences you have sort of
% laboratory automation and you have high-throughput sequencing. In
% oceanography, the sensors are getting cheaper. The models, thanks to
% Moore's law and advances in computing, the simulations you can run are
% getting bigger and higher resolution and, therefore, producing larger and
% larger amounts of data, and so on. And so the rate at which data can be
% produced has far outpaced the rate that we can analyze it or come up with
% the questions we need to ask about it. Okay. The cost of finding,
% integrating and analyzing the data, and then communicating the results to
% others, is the new bottleneck. And this story should sound very similar
% to what we've been saying data science is all about.

\begin{frame}
  \frametitle{eScience, Data Science, 4º paradigma}
  \begin{itemize}
  \item Tradicionalmente, la ciencia se desarrollaba de forma {\bf
      empírica}, por observación, o reproduciendo condiciones en el
    laboratorio
  \item Desde hace unos cientos de años, los {\bf modelos teóricos} también
    se han aceptado como una forma de explicar sucesos, y sugerir
    nuevos experimentos
  \item En los últimos \~{}50~años, {\bf la simulación} se ha usado para
    reproducir condiciones especiales o no reproducibles. Para los modelos
    teóricos que eran demasiado complejos para resolverlos analíticamente,
    se parte de un estado inicial y se comprueba a dónde se ha llegado
  \end{itemize}
\end{frame}

\begin{frame}
  \frametitle{4º paradigma (ii)}
\vspace*{-.5em}
  \begin{itemize}
  \item Hoy: {\bf exploración basada en datos}
    \begin{itemize}
    \item Unifica teoría, experimentación y simulación
\item Los datos se capturan por instrumentos o bien se generan por
  simulaciones
\item Son procesados por software
  \item La información (y el conocimiento extraído) se almacenan en un
    ordenador
    \item Los científicos analizan los ficheros/bases de datos usando
     nuevas herramientas estadísticas y de bases de datos capaces de
     gestionar cada vez más datos (algoritmos distribuidos, NoSQL, etc.)
   \item En vez de ``{\bf preguntar al mundo}'', se obtienen resultados de
     combinar conjuntos ``{\bf descargados}'' de datos de maneras no
     previstas anteriormente
    \end{itemize}
  \item  {\bf Jim Gray} ({\em The Fourth Paradigm}, Microsoft
    Research, 2009)
\end{itemize}
\end{frame}

\begin{frame}
  \frametitle{Huracán Sandy, 2012}
  \framesubtitle{\url{http://rpubs.com/JoFrhwld/sandy} (Howe, 2013)}
\includegraphics[width=\textwidth]{img/sandy}
\end{frame}

\begin{frame}
  \frametitle{¿Cuándo introdujo Apple el ``Swift''?}
  \framesubtitle{Una mirada al tag ``swift'' de todas las preguntas de
    Stackoverflow}
\includegraphics[width=\textwidth]{img/swift-tag}
\end{frame}

\section{Aplicación de Big Data a distintos ámbitos}

\begin{frame}
  \frametitle{Big Data -- Casos de éxito}
  \framesubtitle{Bernard Marr -- Big Data in Practice}
\centering\includegraphics[height=.8\textheight]{img/marr-book}
\end{frame}

\begin{frame}[allowframebreaks]
  \frametitle{Big Data}
  \begin{itemize}
  \item Cuarto paradigma
  \item Ejemplos de big data tomando fuentes dispares e inicialmente no
    conectadas
  \item Campos de aplicación: desde empresas hasta investigación biomédica,
    IoT
  \item Muchos más datos, algoritmos mejorados, hardware mejorado, etc.
    Machine Learning, Data Science
  \item Escalabilidad, concurrencia, distribución
  \item Map-Reduce
  \item Ecosistema Apache: Zookeeper, Hadoop, HBase, Spark, Hive, Impala
  \item Bases de datos NoSQL
  \item Arquitectura Lambda
  \end{itemize}
\end{frame}

\section{Introducción a los sistemas NoSQL}

\begin{frame}
  \frametitle{NoSQL}
\begin{itemize}
% \item Sobre los \~{}2010s, se renueva la búsqueda de la escalabilidad, con
%   el abaratamiento del {\em hardware}
% \item Se {\bf diversifican los problemas}, la inclusión del {\bf análisis
%     de todos los datos disponibles por parte de las empresas}
% \begin{itemize}
% \item incluso de algunos que no se había pensado usar, p. ej. {\em
%     clickstreams\/}), {\bf publicidad a la carta}, etc.
% \end{itemize}
\item {\bf NoSQL} \ra{} {\em hashtag\/} llamativo que se
  eligió para una conferencia en~2009 (Johan Oskarsson de Last.fm)
\item Ahora se asocia a cientos de bases de datos diferentes,
  que se han clasificado en varios tipos (las veremos después),
  caracterizadas por {\bf no usar SQL} como modelo de datos
\item {\bf NoSQL} \ra{} {\bfseries\itshape Not Only SQL} (no sólo SQL)
% \item Big Data \ra{} hay una variedad de fuentes de datos ({\bf persistencia
%     polígota})
  \end{itemize}
\end{frame}

% \begin{frame}
%   \frametitle{NoSQL}
% \centering\includegraphics[width=\textwidth]{img/nosqldatabases3}
% \end{frame}

% \begin{frame}
%   \frametitle{NoSQL}
% \centering\includegraphics[width=\textwidth]{img/nosqldatabases}
% \end{frame}

% \begin{frame}[allowframebreaks]
%   \frametitle{NoSQL -- ¿Por qué se plantearon?}
% %En general, el desarrollo de NoSQL ha venido motivado, entre otras, por una
% %serie de circunstancias:
% \begin{enumerate}
% \item {\bf Mayor escalabilidad horizontal}
%   \begin{itemize}
%   \item conjuntos de datos muy muy grandes
%   \item sistemas de alto volumen de escrituras ({\em streaming\/} de
%     eventos, aplicaciones sociales)
%   \end{itemize}
% \item {\bf Demanda de productos de software libre} (crecimiento de las {\em
%     start-ups})
% \item {\bf Consultas especializadas} no eficientes en el modelo relacional
%   (JOINs)
% \item {\bf Expresividad, flexibilidad, dinamismo}. Frustración con {\bf
%     restricciones} del modelo relacional
%   % \ra{} ({\bfseries\itshape schemaless})
% %\item Paradigma {\bf funcional}: {\em Map-Reduce}
% \end{enumerate}

% \framebreak

% \includegraphics[width=\textwidth]{img/data-growth}

% \end{frame}


% \begin{frame}[allowframebreaks]
%  \frametitle{NoSQL: Características}
% \begin{itemize}
% \item No se basan en SQL
% \item Modelos de datos más ricos
% \item Orientadas a la {\bf Escalabilidad}
% \item Generalmente no obligan a definir un esquema \ra{}
%   {\itshape\bfseries Schemaless}
% \item Surgidos de la comunidad para solucionar problemas, y muchas de
%   ellas son {\bf libres/{\itshape open source}}
% \item Diseñadas \ra{} {\bf procesamiento distribuido}
% \item Principios funcionales \ra{} {\bf MapReduce}
% \item Generalmente implementan {\bf consistencia relajada}
% \end{itemize}

% \framebreak

% \begin{block}{Categorías de NoSQL}
%     \begin{itemize}
%     \item Bases de datos Key-Value
%     \item Bases de datos Documentales
%     \item Bases de datos columnares
%     \item Bases de datos de grafos
%     \item Bases de datos de arrays
%     \end{itemize}
% \end{block}
%\end{frame}


% \begin{frame}
%   \frametitle{Evolución desde el modelo relacional}
% \begin{itemize}
% \item El {\bf modelo relacional} $\Rightarrow$ {\bf predominante en los
%   últimos~\~{}30~años}
% \item Tiene sus raíces en el denominado {\em business data processing},
%   procesamiento de transacciones y {\em batch}
% \item Propuesto por Codd en los~70, {\bf de alto nivel}
% \item Actualmente los {\bf sistemas SQL están muy optimizados}:
% \begin{itemize}
% \item el {\bf grado de implantación es mayoritario}
% \item para el 99\% de los problemas (que caben en un ordenador) es
%   eficiente y adecuado
% \end{itemize}
% \end{itemize}
% \end{frame}

% % http://db-engines.com/en/ranking/


% \begin{frame}
%   \frametitle{Adopción de NoSQL}
% Twitter cambiando a Cassandra por~2010\\
% Cassandra desarrollada en Facebook en~2009\\
% \includegraphics[width=\textwidth]{img/twitter-cassandra}
% \end{frame}

% \begin{frame}
%   \frametitle{Adopción NoSQL. Ranking julio 2017}
% \framesubtitle{Fuente: \url{http://db-engines.com/en/ranking/}}
% \vspace*{.1ex}
% \centering\includegraphics[width=.8\textwidth]{img/ranking-bbdd}
% \end{frame}

% \begin{frame}
% \frametitle{Adopción NoSQL.Tendencia julio 2017}
% \framesubtitle{Fuente: \url{http://db-engines.com/en/ranking/}}
% \includegraphics[width=\textwidth]{img/nosql-database-ranking}
% \end{frame}

% \begin{frame}
%   \frametitle{Adopción NoSQL. Análisis}
% \begin{alertblock}{Análisis}
% \begin{itemize}
% \item Dominan los grandes SGBDR
% \item El {\em Open Source} tiene una importancia crucial (MySQL,
%   MongoDB, etc.)
% \item Varias bases de datos NoSQL entre las~10 primeras. Muchas en las~20
%   primeras
% \item La distancia entre los grandes SGBDR y el primer NoSQL (MongoDB) es
%   de~5$\times$
% \item Paradigmas más ``atrevidos'' como el de grafos están entre los~20
%   primeros (Neo4j)
% \end{itemize}
%   \end{alertblock}
% \end{frame}



\subsection{La importancia de la escalabilidad}

\pgfdeclareimage[height=2.7em]{sobremesa}{img/server1}%img/servidor}
\pgfdeclareimage[height=.7em]{switch}{img/switch1}%img/switch}

\newsavebox{\network}

\begin{lrbox}{\network}
\begin{tikzpicture}
  \foreach \x in {0,...,5}
    \foreach \y in {0,...,3}
    \node [] (\x\y) at (1.5*\x,1.5*\y) {\pgfuseimage{sobremesa}};

% switch
\node[inner sep=0pt] (switch) at (1.5*2.5, 1.5*4) {\pgfuseimage{switch}};

\begin{scope}[on background layer]
  \foreach \x in {0,...,5}
    \foreach \y in {0,...,3}
      \draw[gray!50] (switch)--(\x\y) ;
\end{scope}
\end{tikzpicture}
\end{lrbox}

\begin{frame}
\frametitle{Cambio de perspectiva: Red}

\begin{overlayarea}{\textwidth}{.8\textheight}
\only<1->{%
\begin{center}
\usebox{\network}
\end{center}%
}%
\only<2>{
\vspace*{-13em}
  \begin{block}{Almacenamiento distribuido}
    \begin{itemize}
    \item Desde los 90's: Clústers/NOC/COW: procesamiento masivamente
      paralelo
\begin{center}
{\color{red}SIN EMBARGO...}
\end{center}
\item Almacenamiento no distribuido
\item Ahora los nodos $\Rightarrow$ también {\bf almacenamiento}
\item Minimizar el verdadero cuello de botella: {\bf trasiego de
    información por la red}
    \end{itemize}

  \end{block}%
}%
\only<3>{
\vspace*{-10em}
\begin{block}{Procesamiento distribuido}
\begin{itemize}
\item Necesidad de {\bf paralelización máxima}
\item {\bf Escalabilidad}
\item Explotación de la {\bf localidad de los datos}:
  \begin{itemize}
  \item Datos producidos en cada nodo se utilizan en siguientes iteraciones
  \item Cada nodo puede hacer de servidor para recibir datos
\end{itemize}
\end{itemize}
\end{block}
}%
\only<4>{
\vspace*{-9em}
\begin{block}{Procesamiento distribuido}
\begin{itemize}
\item Vuelta al modelo funcional inherentemente paralelo: (e.g. {\bf
    Map-Reduce})
\item Almacenamiento distribuido: (e.g. {\bf HDFS})
\item Coordinación distribuida: (e.g. {\bf Zookeeper})
\end{itemize}
\end{block}%
}%
\only<5>{
\vspace*{-13em}
\begin{block}{Modelo de datos}
\begin{itemize}
  \item El modelo relacional limita a tablas con valores primitivos y
    relaciones {\em Primary Key\/}/{\em Foreign Key}
  \item Pero en programación se utilizan {\bf listas}, {\bf arrays}, {\bf
      tipos de datos compuestos} ({\color{red}{\em gap\/} semántico})
  \item ACID es {\bf muy compleja y costosa} en ambientes distribuidos
    (quizá {\bf no necesaria} en algunas aplicaciones)
\end{itemize}
\end{block}%
}%
\only<6>{
\vspace*{-10em}
\begin{block}{Modelo de datos (ii)}
\begin{itemize}
\item ¿Si se pudiera ver como un {\bf GRAN ARRAY}?
\begin{itemize}
\item Cada nodo almacenaría una parte del array
\item Búsqueda aleatoria {\bf muy rápida} (árboles B)
%\item Uso de {\bf filtros de Bloom}
\item Uso de {\bf objetos complejos} (p. ej. {\bf documentos JSON}),
  para mantener la {\bf localidad espacial de datos relacionados} (+
  después)
\item Transacciones limitadas al {\bf objeto complejo}
\end{itemize}
\end{itemize}
\end{block}%
}%
\end{overlayarea}
\end{frame}


% \begin{frame}[allowframebreaks,fragile]
% \frametitle{Map-Reduce}

%  {\bf Map-Reduce} es el principal mecanismo de búsqueda y transformación en
%  BBDD NoSQL. Tiene su origen en {\bf lenguajes funcionales}:
%   \begin{block}{{\tt map()}}
%     Ejecuta una misma función sobre todos los elementos de un conjunto
%   \end{block}
%   \begin{block}{{\tt reduce()}}
%     Procesa un conjunto de valores para producir un valor de salida
%   \end{block}

% \framebreak

% \begin{itemize}

% \item Map-Reduce combina ambas operaciones:
% \begin{itemize}
% \item Una misma operación {\tt map()} a cada dato residente en un nodo se
%   realiza de forma paralela en {\bf todos} los nodos simultáneamente
% \item Con los resultados parciales de cada nodo, una función {\tt reduce()}
%   genera un resultado (o un conjunto de resultados) final
% \item Hay un proceso intermedio de {\em shuffle} para agrupar valores
%   relacionados antes del {\tt reduce()}
% \item Resultados parciales en el mismo nodo (localidad) $\Rightarrow$
%   procesamientos {\bf en cadena}
% \end{itemize}
% \end{itemize}

% % \framebreak

% % \centering\includegraphics[width=.9\textwidth]{img/mapreduce1}

% \framebreak

% \centering\includegraphics[width=\textwidth]{img/MapReduceWordcount}
% (de \url{http://www.milanor.net/blog/an-example-of-mapreduce-with-rmr2/})
% \end{frame}

% \begin{frame}
% \frametitle{Map-Reduce en entornos Big Data}
% \begin{itemize}
%   \item Entrada \ra{} siempre pares $<key,value>$

%   \item {\tt map()} produce otro conjunto de valores
%     $\{<key1,value1>,<key2,value2>,...\}$

% \item {\em Shuffle} agrupa los valores con la misma clave:
% \begin{displaymath}
% \{<key1,\{val1,val3,...\}>,<key2,\{val2,val4,...\}>,...\}
% \end{displaymath}

% \item {\tt reduce()} procesa cada lista de valores con la misma clave, y
%   produce otros elementos $<key',value'>$

% \item Hay procesamientos difíciles de expresar en Map-Reduce $\Rightarrow$
%   varias operaciones M/R {\bf en cadena}
%   \end{itemize}

% \end{frame}


% \begin{frame}[fragile,allowframebreaks]
%   \frametitle{Map-Reduce como generalización de consultas}

%   {\bf Ejemplo}: Imagínese un biólogo marino que hace anotaciones de cada
%   animal que ve en el océano, y quiere saber cuántos tiburones ha visto por
%   mes:

%   \begin{lstlisting}[language=SQL]
% SELECT MONTH(observation_timestamp) AS observation_month,
%        sum(num_animals) AS total_animals
% FROM observations
% WHERE family = 'Sharks'
% GROUP BY observation_month;
% \end{lstlisting}

%   \framebreak

% MongoDB con el API de MapReduce:
% \begin{lstlisting}[language=Javascript,basicstyle=\footnotesize\tt]
% db.observations.mapReduce(
%   function map() {
%     var year = this.observationTimestamp.getFullYear();
%     var month = this.observationTimestamp.getMonth() + 1;
%     emit(year + "-" + month, this.numAnimals);
%   },
%   function reduce(key, values) {
%     return Array.sum(values);
%   },
%   {
%     query: { family: "Sharks" },
%     out: "monthlySharkReport"
%   });
% \end{lstlisting}

% \end{frame}


% \begin{frame}[allowframebreaks]
%   \frametitle{Tecnologías habilitadoras de HBase}
% HBase sobre Hadoop (HDFS y Map-Reduce)
% \begin{itemize}
% \item Una {\bf tabla} está formada por un número de filas, identificadas
%   por una clave
% \item Cada tabla se divide en {\bf regiones} (por rangos de clave ordenada
%   lexicográficamente) ({\bf partición horizontal})
% \item Una instalación de HBase utiliza un conjunto de {\bfseries\itshape
%     Region Servers}: nodos de computación con almacenamiento local (y
%   conectados al clúster HDFS)

% \framebreak

% \item Cada grupo de columnas (llamado {\bfseries\itshape Column~Family\/})
%   se almacena en un {\bf Store} ({\bf partición vertical})
% \item El {\bf Store} tiene una parte en memoria ({\bf MemStore}) y,
%   opcionalmente, un almacenamiento en disco, ({\bf HFile})
% \item Los {\bf HFile} se distribuyen usando HDFS para lograr replicación y
%   tolerancia a fallos
% \end{itemize}
% \end{frame}

% \begin{frame}[allowframebreaks]
%   \frametitle{HDFS}
% HDFS posee una arquitectura distribuida:
%     \begin{itemize}
%     \item {\bf NameNode} -- Almacena información sobre qué partes
%       ({\bfseries\itshape Chunks}) tiene cada fichero, y dónde están
%       almacenadas (y replicadas)
%     \item {\bf Secondary Namenode} -- Sustituto del {\bf NameNode} en caso
%       de fallo
%     \item {\bf DataNode}s -- Almacenan los {\em chunks\/} de cada fichero.
%       Cada {\em chunk} puede estar replicado un número de veces,
%       dependiendo de la configuración de HDFS
%     \item {\bf Zookeeper} -- Se encarga de mantener una consistencia de
%       {\em clúster} (saber qué nodos hay conectados y activos) y
%       sincronización de datos
%     \end{itemize}
%   \includegraphics[width=\textwidth]{img/hdfs1}
% \end{frame}

% \subsection{Bases de Datos de Grafos}

% \begin{frame}[allowframebreaks]
%   \frametitle{Bases de Datos de Grafos}
% \vspace*{-1ex}
%   \begin{itemize}
%   \item Las bases de datos de grafos llevan el mecanismo {\em muchos a
%       muchos} al extremo
%   \item Datos en los que existen muchas relaciones entre sí y {\bf las
%       relaciones} tienen un significado primordial
% \item Las bases de datos de grafos se basan en la construcción y consulta
%   de un grafo que consta de
%   \begin{itemize}
%   \item {\bf Vértices} también llamados {\em nodos} o {\em entidades}, y
%   \item {\bf Aristas} ({\bfseries\itshape Edges}), también llamados {\em
%       relaciones}
%   \end{itemize}
% \item Los grafos pueden capturar relaciones complejas entre
%   entidades y ofrecen lenguajes de búsqueda, actualización y creación que
%   permiten trabajar con subconjuntos del grafo
% % \item Orígenes en las bases de datos de hechos (con lenguajes de consulta
% %   lógicos (p. ej. {\em Datalog})
% \item Origen en las bases de datos de hechos ({\em Datalog\/})
% \item Ejemplos: FlockDB, Neo4J, OrientDB
% \end{itemize}
% \end{frame}

% \begin{frame}[plain]
% \includegraphics[width=\textwidth]{img/graph}
% \end{frame}

% \begin{frame}[fragile,plain]
% (Nota: Usa la sintaxis PostgreSQL para {\tt json})
% \begin{lstlisting}[language=SQL]
% CREATE TABLE vertices (
%   vertex_id integer PRIMARY KEY,
%   properties json
% );

% CREATE TABLE edges (
%   edge_id integer PRIMARY KEY,
%   tail_vertex integer REFERENCES vertices (vertex_id),
%   head_vertex integer REFERENCES vertices (vertex_id),
%   label text,
%   properties json
% );

% CREATE INDEX edges_tails ON edges (tail_vertex);
% CREATE INDEX edges_heads ON edges (head_vertex);
% \end{lstlisting}
% \end{frame}

% \begin{frame}[fragile,allowframebreaks]
%   \frametitle{Ejemplo de datos y consulta en Neo4J}
% \begin{block}{}
% \begin{lstlisting}
% CREATE
%   (NAmerica:Location {name:'North America', type:'continent'}),
%   (USA:Location {name:'United States', type:'country' }),
%   (Idaho:Location {name:'Idaho', type:'state' }),
%   (Lucy:Person {name:'Lucy' }),
%   (Idaho)-[:WITHIN]->(USA)-[:WITHIN]-> (NAmerica),
%   (Lucy) -[:BORN_IN]-> (Idaho)
% \end{lstlisting}
% \end{block}

% \framebreak

% Y de consulta:
% \begin{block}{}
% \begin{lstlisting}
% MATCH
% (person) -[:BORN_IN]-> () -[:WITHIN*0..]-> (us:Location {name:'United States'}),
% (person) -[:LIVES_IN]-> () -[:WITHIN*0..]-> (eu:Location {name:'Europe'})
% RETURN person.name
% \end{lstlisting}
% \end{block}

% (con esta consulta tan cercana al lenguaje natural, estamos buscando los
% emigrantes de EEUU en Europa)

% \end{frame}


% \begin{frame}[allowframebreaks]
%   \frametitle{Aplicabilidad de Grafos}
%   \begin{itemize}
%   \item Los grafos, conceptualmente, aparecen en casi cualquier dominio
%   \item Además, su flexibilidad hace que se puedan aplicar de diferentes
%     formas
%   \item Por ejemplo, una relación de {\em follow\/} entre usuarios:

%     \begin{center}
%       \includegraphics[width=.4\textwidth]{img/graph1}
%     \end{center}
%   \end{itemize}

%     \begin{columns}
%       \begin{column}{.5\textwidth}
%         \begin{itemize}
%       \item (nótese cómo {\bf Billy no ha seguido a Ruth}: las relaciones
%         pueden ser unidireccionales o bidireccionales)

%       \item Incluso se puede usar para guardar el conjunto de mensajes
%         que se intercambian:
%       \end{itemize}
%     \end{column}
%       \begin{column}{.5\textwidth}
%         \begin{center}
%           \includegraphics[width=.8\textwidth]{img/graph2}
%         \end{center}
%       \end{column}
%     \end{columns}

% \end{frame}


% \section{Referencias}

% \begin{frame}[fragile,allowframebreaks]
%   \frametitle{Referencias}

% \begin{thebibliography}{Paternostro, 2009}

% \setbeamertemplate{bibliography item}[book]
% \bibitem[Marz, 2015]{Marz2015}
% Nathan Marz, James Warren
% \newblock {\em Big Data: Principles and best
%   practices of scalable realtime data systems}
% \newblock Manning Publications,~2015

% \bibitem[Redmond, 2012]{Redmond2012}
% Eric Redmond, Jim R. Wilson
% \newblock {\em Seven Databases in Seven Weeks: A Guide to Modern Databases
%   and the NoSQL Movement}
% \newblock Pragmatic  Bookshelf,~2012

% \bibitem[Sadalage, 2013]{Sadalage2013}
% Pramod J. Saldage, Martin Fowler
% \newblock {\em NoSQL Distilled. A Brief Guide to the Emerging World of
%   Polyglot Persistence}
% \newblock Addison-Wesley,~2013

% \bibitem[Wilson, 2012]{Wilson2012}
% Jim R. Wilson, Eric Redmond
% \newblock {\em Seven Databases in Seven Weeks}

% \bibitem[Kleppmann, 2016]{Kleppmann2016}
% Kleppmann
% \newblock  \emph{Designing Data Intensive Applications}

% \bibitem[George, 2011]{George2011}
% Lars George
% \newblock {\em HBase, The Definitive Guide}

% \setbeamertemplate{bibliography item}[online]
% \bibitem[ApacheHBaseTeam, 2016]{ApacheHBaseTeam2016}
%   The Apache HBase Team
%   \newblock {\em Apache HBase Reference Guide}
%   \newblock \url{https://hbase.apache.org/book.html}

% \bibitem[HBaseCon, 2012a]{HBaseCon2012a}
%   Ian Varley
% \newblock Vídeo: {\em HBase Schema Design}
% \newblock
% \url{http://www.cloudera.com/content/dam/www/marketing/resources/events/hbase-con/video-hbasecon-2012-hbasecon-2012.png.landing.html}
% \newblock Transparencias:
% \url{http://es.slideshare.net/ivarley/hbase-schema-design-hbasecon-2012}

% \bibitem[George, 2013]{George2013}
%   Lars George
% \newblock {\em HBase Schema Design}
% \newblock \url{https://2013.nosql-matters.org/cgn/wp-content/uploads/2013/05/HBase-Schema-Design-NoSQL-Matters-April-2013.pdf}
% \newblock Otro vídeo: \url{https://vimeo.com/44715954}
% \newblock Sus transparencias: \url{http://2012.berlinbuzzwords.de/sites/2012.berlinbuzzwords.de/files/slides/hbase-lgeorge-bbuzz12.pdf}

% % \bibitem[Khurana, 2013]{khurana2013}
% % Amandeep Khurana, ;login:
% % \newblock {\em Introduction to HBase Schema Design}
% % \newblock \url{http://0b4af6cdc2f0c5998459-c0245c5c937c5dedcca3f1764ecc9b2f.r43.cf2.rackcdn.com/9353-login1210_khurana.pdf}

% % \bibitem[McDonald, 2015]{mcdonald2015}
% %   Carol McDonald
% %   \newblock {\em Guidelines for HBase Schema Design}
% %   \newblock \url{https://www.mapr.com/blog/guidelines-hbase-schema-design}

% \end{thebibliography}
% \end{frame}


\end{document}

%%% Local variables:
%%% mode: LaTeX
%%% TeX-master: t
%%% ispell-local-dictionary: "spanish"
%%% fill-column: 75
%%% TeX-parse-self: t
%%% TeX-auto-save: t
%%% End:
%%% vim: expandtab shiftwidth=2 tabstop=2
